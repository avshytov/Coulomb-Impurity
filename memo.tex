\documentclass[preprint,aps,prb]{revtex4}
\usepackage{amssymb,amsmath,bm,graphicx}
\begin{document}
\begin{center}
{\LARGE\bf
Polarization operator in graphene: expansion over angular momentum
channels, anomaly, etc}
\\
{\bf A. Shytov}
\end{center}

Goals: analyze contribution of individual momentum channels 
to the screening charge distribution, derive screening
in the presence of AB flux.

\section{Green's function}

Retarded Green's function~$G_{R}^{(m)}(\epsilon, r, r')$
in a given angular momentum channel~$m$ satisfies the equation
\begin{equation}
\left\{ 
\epsilon + i {\hat\sigma_x} \frac{\partial}{\partial r}
+ \frac{i}{r} \left(
\begin{array}{cc}
0 & m + 1\\
-m  & 0
\end{array}
\right)
\right\}
\hat{G}_R^{(m)}(\epsilon, r, r') = \frac{\hat{1}}{2\pi r'}\, \delta (r - r')
\end{equation}
The solution to this one-dimensional problem 
can be sought in the form
\begin{equation}
{\hat G}^{(m)}_{R}(\epsilon, r, r') = 
\left\{ 
\begin{array}{lcr}
u (r) \otimes a (r') & {\rm if} & {r < r'} \\
v (r) \otimes b (r') & {\rm otherwise} &
\end{array}
\right.
\ , 
\end{equation}
where~$u_j(r)$ and~$v_j(r)$ are two solutions of Dirac equation which
satisfy boundary conditions at~$r = 0$ and~$r = \infty$. 
The quantities~$a_j(r')$ and~$b_j(r')$ are to be chosen to satisfy
the boundary condition at~$r = r'$:
\begin{equation}
i \sigma_x \left. G_R^{(m)}(\epsilon, r, r') \right|_{r = r'-0}^{r =
r'+0} = \frac{1}{2\pi r}
\ .
\end{equation}
This gives
\begin{equation}
a_j = c_j v_j(r') \ , \qquad b_j = c_j u_j(r') \ ,  
\end{equation}
with
\begin{equation}
\label{c-eq}
c_1 = - c_2 = \frac{1}{2\pi i r} 
\frac{1}{u_1(r') v_2(r') - u_2(r') v_1(r')}
\ .
\end{equation}

The solution regular at~$r = 0$ is
\begin{equation}
\label{eq-u}
u (r) = \left(
\begin{array}{cc}
J_{m} (|\epsilon| r) \\
i \mathop{\rm sign}\nolimits\epsilon J_{m + 1}(|\epsilon| r)
\end{array}
\right)
\ , 
\end{equation}
where~$J_{m}(x)$ is Bessel function. (From now on, we assume~$m > -1$. 
Otherwise, one has to use~$|m|$ and~$|m + 1|$ instead of~$m$ and~$m +
1$, and change the sign of the second component of~$u$.)
The retarded solution is 
given by Hankel function:
\begin{equation}
\label{eq-v}
v(r) = \left(
\begin{array}{c}
H_{m}^{(1)}(|\epsilon|r) \\
i \mathop{\rm sign}\nolimits\epsilon H_{m + 1}^{(1)}(|\epsilon|r)
\end{array}
\right)
\ .
\end{equation}
The Wronskian in Eq.(\ref{c-eq}) takes the form
\begin{eqnarray}
u_1(r') v_2(r') - u_2(r') v_1(r') &=& i  \mathop{\rm
sign}\nolimits\epsilon \left[
J_m(|\epsilon|r') H_{m + 1}(|\epsilon|r') 
- J_{m + 1}(|\epsilon|r') H_m(|\epsilon|r')
\right] 
\\
\nonumber
&=&  i  \mathop{\rm sign}\nolimits\epsilon 
\left[ J_{m} (|\epsilon|r') Y_{m + 1}(|\epsilon|r') 
- J_{m + 1} (|\epsilon|r') Y_{m}(|\epsilon|r')
\right] \ .
\end{eqnarray}
Using  the relation [Abramowitz, 9.1.16], we find
\begin{equation}
u_1(r') v_2(r') - u_2(r') v_1(r')
= - \frac{2 i}{\pi \epsilon r'}
\ .
\end{equation}
Therefore, retarded Green's function has the form
\begin{equation}
\hat{G}^{(m)}_R(\epsilon, r, r') = \frac{\epsilon}{4} 
\left\{
\begin{array}{ccc}
u(r) \otimes \hat{\sigma}_3 v(r') & {\rm if} & r < r' \\
v(r) \otimes \hat{\sigma}_3 u(r') & {\rm otherwise} &  
\end{array}
\right.
\ , 
\end{equation}
where the spinors~$u(r)$ and~$v(r)$ are given by Eq.(\ref{eq-u}) 
and~(\ref{eq-v}).
(The factor~$\hat{\sigma}_3$ takes into account the relation~$c_1 = -c_2$).

\section{Polarization operator}

The static polarization operator~$\Pi^{(m)}(r, r')$ is related to causal
Green's function~$G^{(m)}(\epsilon, r, r')$ as
\begin{equation}
\Pi^{(m)} = -i \mathop{\rm Tr}\nolimits 
\int\limits_{-\infty}^{\infty} 
\hat{G}^{(m)}(\epsilon, r, r') 
\hat{G}^{(m)}(\epsilon, r', r) \frac{d\epsilon}{2\pi}
\ .
\end{equation}
At zero temperature, the causal Green's function is related to retarded
function~$G_R$ as
\begin{equation}
\hat{G}^{(m)} (\epsilon, r, r') = 
\left\{
   \begin{array}{ccc}
   \hat{G}^{(m)}_R (\epsilon, r, r') & {\rm for} & \epsilon > \epsilon_F \\ 
   \hat{G}^{(m)\ast}_R (\epsilon, r, r') & {\rm for} & \epsilon < \epsilon_F 
   \end{array}
\right.
\ .
\end{equation}
Therefore, one can write the polarization operator as 
\begin{eqnarray}
\Pi^{(m)}(r, r') &=& -i \mathop{\rm Tr}\nolimits 
\left\{
\int\limits_{-\infty}^{\epsilon_F} 
\frac{d\epsilon}{2\pi}
G_R^{(m)\ast} (\epsilon, r, r') G_R^{(m)\ast} (\epsilon, r', r)
+
 \int\limits_{\epsilon_F}^{\infty} \frac{d\epsilon}{2\pi}
 G_R^{(m)} (\epsilon, r, r') G_R^{(m)} (\epsilon, r', r)
 \right\}
\nonumber
\\
 &\equiv& \Pi_n^{(m)} + \Pi_d^{(m)} 
 \ , 
\end{eqnarray}
where
\begin{equation}
\Pi_n^{(m)}(r, r') = 
-i \mathop{\rm Tr}\nolimits 
\int\limits_{-\infty}^{0} \frac{d\epsilon}{2\pi}
G_R^{(m)\ast} (\epsilon, r, r') G_R^{(m)\ast} (\epsilon, r', r)
-i \mathop{\rm Tr}\nolimits 
\int\limits_{0}^{\infty} \frac{d\epsilon}{2\pi}
G_R^{(m)} (\epsilon, r, r') G_R^{(m)} (\epsilon, r', r)
\end{equation}
is the neutral contribution, and
\begin{equation}
\Pi_d^{(m)}(r, r') = -i \mathop{\rm Tr}\nolimits
\int\limits_{0}^{\epsilon_F} \frac{d\epsilon}{2\pi} \left[
G_R^{(m)\ast} (\epsilon, r, r') G_R^{(m)\ast} (\epsilon, r', r)
- G_R^{(m)} (\epsilon, r, r') G_R^{(m)} (\epsilon, r', r)
\right]
\end{equation}
is the doping contribution. The latter can be rewritten as
\begin{eqnarray}
\label{Pi-d}
\Pi_d^{(m)}(r, r' > r) &=& 2 \mathop{\rm Im}\nolimits
\int\limits_{0}^{\epsilon_F} \frac{\epsilon^2 d\epsilon}{32 \pi}
\left[u_1^2(r) - u_2^2(r)\right] \left[v_1^2(r') - v_2^2(r')\right]
\\
&=& \int\limits_{0}^{\epsilon_F} \frac{\epsilon^2 d\epsilon}{8\pi}
\left[J_m^2(\epsilon r) + J_{m + 1}^2(\epsilon r)\right]
\left[J_m(\epsilon r')Y_m(\epsilon r') 
    + J_{m + 1}(\epsilon r')Y_{m + 1}(\epsilon r')\right]
    \ , 
\nonumber
\end{eqnarray}
where~$Y_{m}(\epsilon r)$ is the Bessel function of the second kind, 
$H_m^{(1)}(x) = J_m(x) + i Y_m(x)$.

The neutral contribution~$\Pi_n^{(m)}(r, r')$ is given by a similar 
expression. However, it is convenient to rewrite it by rotating
the integration contour, as this removes oscillating behaviour 
at~$\epsilon = \infty$. (The contour can be rotated because 
Hankel function decay overweights Bessel function increase, due to 
$r' > r$.)
Using~$J_m(iz) = i^{-m} I_m(z)$, 
$H^{(1)}_m(iz) = \frac{2}{\pi} i^{- m - 1} K_m(z)$, 
we find
\begin{equation}
\label{Pi-n}
\Pi_n^{(m)}(r, r' > r) 
= - \int\limits_{0}^{\infty}
    \frac{\epsilon^2 d\epsilon}{4\pi^3} 
    \left[I_m^2(\epsilon r') - I^2_{m + 1}(\epsilon r') \right]
    \left[K_m^2(\epsilon r') - K^2_{m + 1}(\epsilon r') \right]
\ .   
\end{equation}

\section{Anomaly and magic relation}

Let us analyze the properties of neutral polarization
operator~$\Pi_n^{(m)}(r, r')$. At zero Fermi energy, there is no 
intrinsic scale in the problem. Therefore, $\Pi_n^{(m)}(r, r')$
can be represented as a function of $\rho = r/r'$ only: 
\begin{equation}
\Pi_n^{(m)}(r, r') = \frac{1}{4\pi^3} \left\{
\begin{array}{ccc}
\frac{1}{r'^3} {\cal P}^{(m)}\left(\frac{r}{r'}\right) & {\rm for} & r < r' \\
\frac{1}{r^3}  {\cal P}^{(m)}\left(\frac{r'}{r}\right) & {\rm otherwise} &  
\end{array}
\right.
\ , 
\end{equation}
with
\begin{equation}
{\cal P}^{(m)}(\rho) 
=  \int\limits_{0}^{\infty}
    x^2 dx
    \left[I_m^2(x\rho) - I^2_{m + 1}(x\rho) \right]
    \left[K_{m + 1}^2(x) - K^2_{m}(x) \right]
\ .   
\end{equation}
Consider responses to two potentials: $U_1(r) = 1$ and  $U_2(r) = 1/r$. 
In the first case, we 
find, after summing up the~$r < r'$ and $r > r'$ contributions: 
\begin{equation}
\label{rho-const}
\rho^{(m)}_{1, {\rm naive}}(r) = \frac{1}{2\pi^2 r} \int\limits_{0}^{1} 
(\rho + 1) {\cal P}^{(m)}(\rho) d\rho = \frac{1}{2\pi^2 r}
\ .
\end{equation}
In the second case, we find a similar expression:
\begin{equation}
\label{rho-coul}
\rho^{(m)}_{2, {\rm naive}}(r) = \frac{1}{2\pi^2 r^2} \int\limits_{0}^{1} 
(1 + \rho) {\cal P}^{(m)}(\rho) d\rho = \frac{1}{2\pi^2 r^2}
\ .
\end{equation}
Here we used the fact that  both responses involve the same quantity, 
which obeys the ``magic'' identity:
\begin{equation}
\label{magic}
 \int\limits_{0}^{1}
 (1 + \rho) {\cal P}^{(m)}(\rho) d\rho = 1
 \ .
\end{equation}
Eq.~(\ref{magic}) holds
for all values of angular momentum~$m$, both integer and 
non-integer, except for~$m = -1/2$.  
In the latter case,  the response vanishes, 
due to~$K_{-1/2}^2(x) = K_{1/2}^2(x)$. 
(The singular limit of~$m = -1/2$ is discussed further,
see Eq.~(\ref{minus-half}) in the next Section.)
The proof of the identity is given below in Section~\ref{sec:magic}. 


The results~(\ref{rho-const}) and~(\ref{rho-coul}) 
cannot be correct physically. First, the response to a
constant potential must vanish, due to vanishing density of states 
near the neutrality point. Secondly, the response to a $1/r$ potential
is known to vanish at all finite distances, due to dielectric screening
in graphene. 

To resolve this paradox, it is instructive to consider the case of~$m
= -1/2$, when the response function~$\Pi_n^{(-1/2)}$ vanishes
identically. One can show that in this channel Dirac equation reduces
to the equation for two uncoupled fermion branches in the potential~$U(r)$, 
and therefore 
admits the solution in which the potential changes only the phase of 
the wave function, without any effect on the electron density. 
This situation is akin to what happens in a one-dimensional Fermi gas
with a linearized spectrum. In the latter case, it is known that the
density response is determined by chiral anomaly, which results from 
the shift of the kinetic energy cutoff by~$U(r)$. This gives an extra density 
contribution
\begin{equation}
\delta \rho_{\rm anom}^{(m)}(r)  = - \frac{U(r)}{2\pi^2 r}
\ .
\end{equation}
Obviously, such a shift should be applied to all angular momentum
channels, and therefore one has to add this anomalous contribution
to all angular momentum channels. (Indeed, near the large band cutoff
energy the centrifugal potential~$m^2/r^2$ can be neglected.
Therefore, anomalous contribution can be expected to be $m$-independent.) 
This gives for the density
\begin{equation}
\delta \rho^{(m)}(r) = - \frac{U(r)}{2\pi^2 r} 
+ 2\pi \int\limits_{0}^{\infty} 
  \left[\Pi_n^{(m)}(r, r') + \Pi_d^{(m)}(r, r')\right] r' U(r') dr' 
\ , 
\end{equation}
with~$\Pi_n^{(m)}(r, r')$ and~$\Pi_d^{(m)}(r, r')$ given by
Eqs.(\ref{Pi-n}) and~(\ref{Pi-d}) respectively. For~$m \neq -1/2$, 
the anomalous contribution can be combined with regular one
with the help of Eq.(\ref{magic}):
\begin{equation}
\delta \rho^{(m)}(r) = 
 2\pi \int\limits_{0}^{\infty} 
  \left\{ \Pi_n^{(m)}(r, r') \left[ r'U(r') - rU(r) \right] 
  + \Pi_d^{(m)}(r, r')r' U(r')\right\}  dr' 
\ .
\end{equation}
For half-integer momenta, the anomalous contribution of the~$m = -1/2$
channel has to be added explicitly.

\section{Fractional momenta}

The solution given above is strictly correct in the absence of 
Aharonov-Bohm flux tube, which corresponds to integer values of~$m$. 
In the prepense of flux tube, one has to take into account 
magnetic polarization at the origin. However, for thin flux tubes
this effect can be neglected. Indeed, the solution~$u(r)$, regular near the origin
is proportional to~$(\epsilon r)^|m + 1|$ or~$(\epsilon r)^{|m|}$, and 
is therefore suppressed at~$\epsilon a \ll 1$, where~$a$ is the size
of the flux tube. At large energies, $\epsilon a \gg |m|$, the effect of
flux tube is negligible, as can be shown by matching two semiclassical
solutions inside and outside the tube. Thus, the prepense of the tube
would have no significant effect on the treatment of anomaly. 
Note also  that in a neutral graphene the response at distance~$r$ 
is determined by~$\epsilon \sim r^{-1}$. Thus, the flux tube 
mostly affects the screening within itself. (For thick flux tubes,
such as those caused by superconducting vortices, 
one has to use integer momentum screening inside the tube.)

{\bf To be expanded?}

\section{Serial expansion of~${\cal P}^{(m)}(\rho)$, and its Mellin
transform}

It is possible to obtain a serial expansion of the scale-invariant
function~${\cal P}^{(m)}(\rho)$. To simplify the integrand, we write
\begin{equation}
I_m^2(x) = \int\limits_{0}^{2\pi} I_{2m} \left(2 x \cos\theta \right)
\frac{d\theta}{2\pi}
\ , 
\end{equation}
and
\begin{equation}
K_m^2(y) = \frac{1}{2}
\int\limits_0^{\infty} K_{2m}(2 y  \rho \cosh u) du
\ .
\end{equation}
Then, one can employ the recurrence relation between cylindrical
functions:
\begin{eqnarray}
I_{2m}(z) - I_{2m + 2}(z) &=& \frac{2(2m + 1)}{z} I_{2m}(z)
\ , \\
K_{2m}(z) - K_{2m + 2}(z) &=& - \frac{2(2m + 1)}{z} K_{2m + 1}(z)
\ .
\end{eqnarray}
This gives
\begin{equation}
{\cal P}^{(m)}(\rho) = \frac{(2m + 1)^2}{4}\int\limits_{0}^{\infty} dx
\int\limits_{0}^{\infty} du \int\limits_{0}^{2\pi} \frac{d\theta}{2\pi}
\frac{I_{2m + 1}\left(\frac{x \rho \cos \theta}{\cosh u}\right)}{\rho
\cos \theta \cosh^2 u}
K_{2m + 1} (x)
\ .
\end{equation}
(The integration variable~$x$ was rescaled here.) Using the serial expansion
\begin{equation}
I_{2m + 1}(z) = \sum_{k = 0}^{\infty} \frac{\left(\frac{z}{2}
\right)^{2m + 1 + 2k}}{k! \Gamma(k + 2m + 2)}
\ , 
\end{equation}
we find
\begin{equation}
{\cal P}^{(m)}(\rho) = 
\sum_{k = 0}^{\infty} \frac{\rho^{2m + 2k}}{2^{2m + 1  + 2k}} 
\int\limits_{0}^{2\pi} (\cos\theta)^{2m + 2k} \frac{d\theta}{2\pi}
\int\limits_{0}^{\infty} \frac{du}{(\cosh u)^{(2m + 2k + 3}} 
\int\limits_{0}^{\infty} dx x^{2m + 2k + 1} K_{2m + 1}(x)
\ .
\end{equation}
Using the relations
\begin{equation}
\int\limits_{0}^{2\pi} \cos^{n}\theta \frac{d\theta}{2\pi} 
= \frac{\Gamma\left(\frac{n + 1}{2}\right)}{
      \sqrt{\pi} \Gamma\left(\frac{n}{2} + 1\right)}
      \ , \qquad
\int\limits_{0}^{\infty} \frac{du}{\cosh^{n} u} = 
\frac{\sqrt{\pi} \Gamma\left(\frac{n}{2} - 1\right)}{ 
      \Gamma\left(\frac{n - 1}{2}\right)}
\ ,       
\end{equation}
and
\begin{eqnarray}
\int\limits_{0}^{\infty} x^{2k + 2m + 1} K_{2m + 1}(x) dx &=& \frac{1}{2}
\int\limits_{0}^{\infty}dx
\int\limits_{-\infty}^{\infty}dt e^{-x \cosh t} e^{-(2m + 1)t} x^{2k + 2m
+ 1}
\\
\nonumber
 &=& \frac{\Gamma(2k + 2m + 2)}{2} \int\limits_{-\infty}^{\infty}
 \frac{e^{-(2m + 1)t} dt}{\cosh^{2k + 2m + 2}t}  
\\
\nonumber
&=& 2^{2k + 2m}  
 \Gamma\left(k + 2m + \frac{3}{2}\right) \Gamma\left(k + \frac{1}{2}\right)
\ , 
\end{eqnarray}
we finally arrive at the serial expansion for the polarization operator
\begin{equation}
\label{serial}
{\cal P}^{(m)}(\rho) 
= \frac{(2m + 1)^2}{4}\sum_{k = 0}^{\infty} 
\frac{\Gamma\left(k + \frac{1}{2}\right) \Gamma\left(k + m + \frac{1}{2}\right)
      \Gamma\left(k + m + \frac{3}{2}\right) 
      \Gamma\left(k + 2m + \frac{3}{2}\right)
}{
\Gamma\left(k + 1\right) \Gamma\left(k + m + 1\right)
      \Gamma\left(k + m + 2\right) \Gamma\left(k + 2m + 2\right)
}
\rho^{2k + 2m} 
\ . 
\end{equation}
(Note that each gamma function argument  in the denominator is shifted by
1/2 from the one in the numerator. This provides~$1/k^2$ falloff of the
coefficients.) 

One can use this expansion to obtain the response function at~$m
\approx -1/2$. Indeed, this response is dominated by the first term
in the series. Writing~$m = -1/2 + \mu$, and expanding over~$ \mu \ll 1$, 
one finds
\begin{equation}
\label{minus-half}
P^{\left(-\frac{1}{2} + \mu\right)}(\rho) = 2 \mu \rho^{-1 + 2\mu} + O(\mu^2)
\ .
\end{equation}
Then, one can check that the ``magic'' relation~(\ref{magic}) indeed 
holds in this limit. 
One can also get some 
insight about the singular limit of~$m = -1/2$: 
the response becomes more and more
smeared as~$m$ approaches the value of~$1/2$, so that the total integral
in Eq.~(\ref{magic})
remains unchanged. 

The serial expansion can be also used to calculate Mellin transform of
the response. Let us calculate the response to a potential~$U = 1/r^{2z}$. 
By adding the~$r < r'$ and~$r > r'$ contributions, we write
\begin{eqnarray}
\rho^{(m)}(r) &=& \frac{1}{2\pi^2 r^{1 + 2z}} \int\limits_{0}^{1} 
{\cal P}^{(m)}(\rho) (\rho^{2z} + \rho^{1 - 2z}) d\rho 
\nonumber
\\
&\equiv& \frac{1}{2\pi r^{1 + 2z}} P^{(m)}(z)
\ .
\end{eqnarray}
Integrating the serial expansion~(\ref{serial}), we obtain the 
expansion of~$P(z)$ over its poles:
\begin{eqnarray}
P^{(m)}(z) &=& \frac{(2m + 1)^2}{8} 
\sum\limits_{k = 0}^{\infty}
\frac{\Gamma\left(k + \frac{1}{2}\right) \Gamma\left(k + m + \frac{1}{2}\right)
      \Gamma\left(k + m + \frac{3}{2}\right) 
      \Gamma\left(k + 2m + \frac{3}{2}\right)
}{
\Gamma\left(k + 1\right) \Gamma\left(k + m + 1\right)
      \Gamma\left(k + m + 2\right) \Gamma\left(k + 2m + 2\right)
} 
\nonumber
\\
&\times&
\left( \frac{1}{z + m + k + \frac{1}{2}} + \frac{1}{ - z + m + k + 1}   \right)
\ .
\end{eqnarray}
The function~$P(z)$ has an obvious symmetry~$z \to \frac{1}{2} - z$. 
There are two equidistant sequence of poles: $z_k = -m - k - \frac{1}{2}$,  
and~$z_k = m + k + 1$, forming a mirror-symmetric pattern around~$z = \frac{1}{4}$.
Since the  residues fall off as~$1/k^2$, it should have no other 
singularities,  and decay as~$P(z) \sim z^{-2}$ at large~$z$ (away from 
the real axis). The magic relation~(\ref{magic}) is equivalent 
to~$P(0) = P\left(\frac{1}{2}\right) = 1$, see the proof in the
next Section.

It is very tempting to attempt to sum up the series by finding the
function with appropriate residues at the poles. The function
\begin{eqnarray}
P^{(m)}_{1}(z) &=&  
\frac{\Gamma\left(-m - z\right)}{\Gamma\left(\frac{1}{2}  - m - z\right)} 
\frac{\Gamma\left(m + 1 - z \right)}{ \Gamma\left(m + \frac{3}{2} - z\right)}
\frac{\Gamma\left(-z\right)}{\Gamma\left(\frac{1}{2} - z\right)}
\frac{\Gamma\left(1 - z\right)}{\Gamma\left(\frac{3}{2} - z\right)} 
\nonumber
\\
&\times& \tan \pi z \tan \pi (z + m)
\end{eqnarray}
has required residues, at the poles, up to a constant factor. 
However, it also has extra poles
at~$z_k = k + 1$ and~$z_k = -\frac{1}{2} - k$, and therefore cannot
represent~$P(z)$. The extra poles can be eliminated by
multiplying~$P^{(m)}_1(z)$ by~$\sin^2 \pi z$. Such function would 
have correct residues and no extra poles. However, it would also be 
proportional to~$\sin^2 \pi z$ at large~$z$ and therefore can be
exponentially large at the imaginary axis, which rules such a solution
out. The search for the analytical continuation continues. 

It is possible to sum up the series for half-integer momenta. 
In this case, the two pole sequences eat each other. For example, the 
polarization operator in a subcritical~$m = 1/2$ channel is 
\begin{equation}
P^{(m = \frac{1}{2})}(z) 
= - \frac{1}{2}
  \frac{\psi(z + 1) + \psi\left(\frac{3}{2} - z\right) - \psi(2) 
        - \psi\left(\frac{1}{2}\right)}{\left(z - 1 \right)\left(z +
	\frac{1}{2}\right)}
\ , 
\end{equation}
where~$\psi(x) = \Gamma'(x) / \Gamma(x)$ is the digamma function. 
Again, the magic relation holds: $P^{(\frac{1}{2})}(0) = 1$.

For future reference, we also give the expression for~$m = 3/2$, 
\begin{eqnarray}
P^{(m = \frac{3}{2})}(z) 
 &=& - 2
  \left\{
    \frac{2}{5} \frac{1}{\left(z + \frac{1}{2}\right)\left(z - 1\right)}
    \left[
       \psi\left(2 + z\right) 
       + \psi\left(\frac{5}{2} - z\right) 
       - \psi\left(3\right) 
       - \psi\left(\frac{3}{2}\right) 
    \right]
  \right.
\nonumber
\\
  &+&
  \left.
    \frac{3}{5} \frac{1}{\left(z + \frac{3}{2}\right)\left(z - 2\right)}
    \left[
       \psi\left(2 + z\right) 
       + \psi\left(\frac{5}{2} - z\right) 
       - \psi\left(4\right) 
       - \psi\left(\frac{1}{2}\right) 
    \right]
  \right\}
\ , 
\end{eqnarray}
and~$m = 5/2$:
\begin{eqnarray}
P^{(m = \frac{5}{2})}(z) 
 &=& - \frac{9}{2} 
  \left\{
    \frac{9}{35} \frac{1}{\left(z + \frac{1}{2}\right)\left(z - 1\right)}
    \left[
       \psi\left(3 + z\right) 
       + \psi\left(\frac{7}{2} - z\right) 
       - \psi\left(4\right) 
       - \psi\left(\frac{5}{2}\right) 
    \right]
  \right.
\nonumber
\\
  &+&
    \frac{4}{15} \frac{1}{\left(z + \frac{3}{2}\right)\left(z - 2\right)}
    \left[
       \psi\left(3 + z\right) 
       + \psi\left(\frac{7}{2} - z\right) 
       - \psi\left(5\right) 
       - \psi\left(\frac{3}{2}\right) 
    \right]
\\
\nonumber
  &+&
  \left.
    \frac{10}{21} \frac{1}{\left(z + \frac{5}{2}\right)\left(z - 3\right)}
    \left[
       \psi\left(3 + z\right) 
       + \psi\left(\frac{7}{2} - z\right) 
       - \psi\left(6\right) 
       - \psi\left(\frac{1}{2}\right) 
    \right]
  \right\}
\ .
\end{eqnarray}

Another easy case is the one of large angular momentum, $m \gg 1$.
In this case, the sum is determined by its terms with~$k \sim m$. 
At large values of~$k$, 
the ratio of two gamma functions can be approximated as
~$\Gamma(x)/\Gamma\left(x + \frac{1}{2}\right) \approx x^{-1/2}$.
Replacing the sum by the integral and introducing the 
new variable, $q = k + m$, we write
\begin{equation}
P^{(m \gg 1)}(z) = m^2 \int\limits_{m}^{\infty}
\frac{dq}{q\sqrt{q^2 - m^2}} \frac{q}{q^2 - z^2}
\ .
\end{equation}
A trivial integration yields
\begin{equation}
P^{(m \gg 1)}(z) = \frac{m^2}{2iz \sqrt{m^2 - z^2}} 
\ln\frac{1 + i z \sqrt{m^2 - z^2}}{1 - iz \sqrt{m^2 - z^2}}
\ .
\end{equation}
The square root singularity at~$z > m$ is due to merging poles. 
Again, the magic identity~(\ref{magic}) holds: $P^{(m \gg 1)}(0) = 1$.

\section{Proof of the magic identity}
\label{sec:magic}
The expansion~(\ref{serial}) allows one to prove the magic identity.
Let
\begin{equation}
C_k = 
\frac{\Gamma\left(k + \frac{1}{2}\right) \Gamma\left(k + m + \frac{1}{2}\right)
      \Gamma\left(k + m + \frac{3}{2}\right) 
      \Gamma\left(k + 2m + \frac{3}{2}\right)
}{
\Gamma\left(k + 1\right) \Gamma\left(k + m + 1\right)
      \Gamma\left(k + m + 2\right) \Gamma\left(k + 2m + 2\right)
} 
\ .
\end{equation}
Consider  partial sums
\begin{eqnarray}
S_n = 
\sum\limits_{k = 0}^{n}
C_k
\left( \frac{1}{m + k + \frac{1}{2}} + \frac{1}{m + k + 1}   \right)
\end{eqnarray}
seeking~$S_n$ in the form
\begin{equation}
\label{r-ansatz}
S_n = R_n C_n
\ .
\end{equation}
Since~$S_n = S_{n - 1} + C_n$, and
\begin{equation}
\frac{C_n}{C_{n - 1}} = \frac{n (n + m) (n + m + 1) (n + 2m + 1)}{
\left(n - \frac{1}{2}\right)
\left(n + m - \frac{1}{2}\right)
\left(n + m + \frac{1}{2}\right)
\left(n + 2m + \frac{1}{2}\right)
}
\ , 
\end{equation} 
we can write recurrence relation for~$R_n$:
\begin{equation}
\label{recurrent}
R_n = R_{n - 1} \frac{n (n + m) (n + m + 1) (n + 2m + 1)}{
\left(n - \frac{1}{2}\right)
\left(n + m - \frac{1}{2}\right)
\left(n + m + \frac{1}{2}\right)
\left(n + 2m + \frac{1}{2}\right)
} + 
\frac{1}{n + m + \frac{1}{2}} + \frac{1}{n + m + 1}
\ .
\end{equation}
To find a solution to this equation, we introduce a new 
variable 
\begin{equation}
\label{u-ansatz}
R_n = u_n \frac{\left(n + \frac{1}{2}\right)
\left(n + m + \frac{1}{2}\right)
\left(n + 2 m + \frac{3}{2}\right)
}{n + m + 1}
\ .
\end{equation}
Such an ansatz removes three factors in the denominator.
The recurrence equation~(\ref{recurrent}) then takes the form
\begin{eqnarray}
\label{recurrent-u}
\left(n + \frac{1}{2}\right)
\left(n + m + \frac{1}{2}\right)^2
\left(n + 2 m + \frac{3}{2}\right) u_n
&-& 
n 
\left(n + m + 1\right)^2
\left(n + 2 m + 1\right) 
u_{n - 1}
\nonumber
\\
&=& 2n + 2m + \frac{3}{2}
\ .
\end{eqnarray}
Interestingly, this equation admits a constant solution, $u_n = u_0$. 
Indeed, 
a straightforward calculation gives for the l.h.s.
\begin{eqnarray}
\left(n + \frac{1}{2}\right)
\left(n + m + \frac{1}{2}\right)^2
\left(n + 2 m + \frac{3}{2}\right)
&-& n 
\left(n + m + 1\right)^2
\left(n + 2 m + 1\right)
\\
\nonumber
 &=& \left(m + \frac{1}{2}\right)^2 \left( n + m + \frac{3}{4}\right)
 \ , 
\end{eqnarray}
which coincides with the r.h.s. of Eq.(\ref{recurrent-u}) 
up to a constant factor. 
Thus, 
\begin{equation}
u_n = u_0 = \frac{2}{\left(m + \frac{1}{2}\right)^2}
\end{equation}
is a solution of Eq.~(\ref{recurrent-u}). One can also check that
this solution satisfies the initial condition at~$n = 0$. Indeed, 
the partial sum~$S_0$ is simply  
\begin{equation}
S_0 = C_0 \left(\frac{1}{m + \frac{1}{2}} + \frac{1}{m + 1}\right)
= C_0 \frac{2 \left(m + \frac{3}{4}\right)}{\left(m + \frac{1}{2}\right)
\left(m + 1\right)}
\ .
\end{equation}
On the other hand, the ansatz~(\ref{r-ansatz}), (\ref{u-ansatz}) gives
\begin{equation}
S_0 = C_0 R_0 = C_0 u_0 
\frac{\frac{1}{2}\cdot\left(m + \frac{1}{2}\right) 
\left(2m + \frac{3}{2}\right)}{\left(m + 1\right)}
= C_0 \frac{2 \left(m + \frac{3}{4}\right)}{\left(m + \frac{1}{2}\right)
\left(m + 1\right)} 
\ .
\end{equation}
Hence, we see that  
\begin{eqnarray}
\label{identity-sum}
S &=& \sum_{k = 0}^{\infty} C_k 
C_k
\left( \frac{1}{m + k + \frac{1}{2}} + \frac{1}{m + k + 1}   \right) 
\\
\nonumber
&=& 
\lim_{n \to \infty} u_0 C_n \frac{\left(n + \frac{1}{2}\right)
\left(n + m + \frac{1}{2}\right)
\left(n + 2 m + \frac{3}{2}\right)}{n + m + 1} 
\nonumber
\\
&=& \lim_{n \to \infty }u_0 n^2 C_n = u_0  
= \frac{2}{\left(m + \frac{1}{2}\right)^2}
\ .
\end{eqnarray}
Evaluating the left-hand side of the 
magic identity~(\ref{magic}) with the help of serial
expansion~(\ref{serial}) and employing the
relation~(\ref{identity-sum}), one can 
show that the magic identity holds for arbitrary values of~$m$. 

\section{Second-order correction}

To properly treat the band contribution to a better accuracy, one can
add the second correction to the RPA analysis. Here we present
the calculation of this correction without resolving the contribution
of individual angular momentum channels. 

By perturbation theory, the correction can be found as
\begin{equation}
\label{U2-initial}
\delta \rho^{(2)}({\bf r})
= -\mathop{\rm Im}\nolimits \int \frac{d\epsilon}{2\pi}
\int d^2 r_1 d^2 r_2 
\mathop{\rm Tr}\nolimits G(\epsilon, {\bf r} - {\bf r}_1 ) 
G(\epsilon, {\bf r} - {\bf r}_1) G(\epsilon, {\bf r}_1 - {\bf r}_2)
G(\epsilon, {\bf r}_2 - {\bf r}) U({\bf r}_1) U({\bf r}_2)
\ , 
\end{equation}
where~$G$ is the causal Green's function. 
In Fourier space, this can be cast in the form
\begin{equation}
\delta \rho^{(2)} ({\bf q}) = \int U({\bf q}_1) U({\bf q}_2) 
\delta({\bf q} - {\bf q}_1 - {\bf q}_2) \Sigma ({\bf q}_1, {\bf q}_2)
\frac{d^2{\bf q}_1}{(2\pi)^2}
\frac{d^2{\bf q}_2}{(2\pi)^2}
\ .
\end{equation}
The kernel~$\Sigma({\bf q}_1, {\bf q}_2)$ is given by 
\begin{equation}
\label{U2-momentum}
\Sigma({\bf q}_1, {\bf q}_2) 
=  \mathop{\rm Im}
   \int\limits_{-\infty}^{\infty} \frac{d\epsilon}{2\pi}
   \int\frac{d^2{\bf p}}{(2\pi)^2}
   \mathop{\rm Tr}\nolimits G(\epsilon, {\bf p}) G(\epsilon, {\bf p} +
   {\bf q}_1) G(\epsilon, {\bf p} + {\bf q}_2)
\ , 
\end{equation}
The Green's function in the momentum space is
\begin{equation}
G(\epsilon, {\bf p}) = \frac{1}{{\tilde\epsilon} - {\bm \sigma} \cdot {\bf p}}
= \frac{{\tilde\epsilon} + {\bm \sigma} \cdot {\bf p}}{
   {\tilde\epsilon}^2 - {\bf p}^2
   }
   \ , 
\end{equation}
where~$\tilde{\epsilon} = \epsilon + \mu + i 0 \mathop{\rm sgn}
\epsilon$
takes care of the position of Fermi level~$\mu$ and Fermi statistics. 

Calculation of the trace in Eq.(\ref{U2-momentum}) leads to the 
following expression: 
\begin{equation}
\mathop{\rm Tr}\nolimits (\tilde\epsilon + {\bm \sigma} \cdot {\bf p})
 \left[\tilde\epsilon + {\bm \sigma} \cdot ({\bf p} + {\bf q}_1) \right]
 \left[\tilde\epsilon + {\bm \sigma} \cdot ({\bf p} + {\bf q}_2) \right]
 \ , 
\end{equation}
which can be found by using the standard identities:
\begin{equation}
\mathop{\rm Tr}\nolimits {\bm \sigma} \cdot {\bf a} = 0
\ , \qquad \mathop{\rm Tr}\nolimits ({\bm \sigma} \cdot {\bf a} )
({\bm \sigma} \cdot {\bf b} ) = 2 {\bf a} \cdot {\bf b}, 
\ , \qquad \mathop{\rm Tr}\nolimits ({\bm \sigma} \cdot {\bf a} )
({\bm \sigma} \cdot {\bf b} ) ({\bm \sigma} \cdot {\bf c} ) = 
2 i {\bf a} \cdot [{\bf b} \times {\bf c}]
\ .
\end{equation}
This readily yieds
\begin{equation}
\label{U2-trace}
\Sigma({\bf q}_1, {\bf q}_2) 
= -2 \mathop{\rm Im}\nolimits \int\limits_{-\infty}^{\infty}
\frac{d\epsilon}{2\pi}
\int\frac{d^2{\bf p}}{(2\pi)^2} 
\frac{
  \tilde{\epsilon}
  \left[\tilde{\epsilon}^2 
     + {\bf p}\cdot({\bf p} + {\bf q}_1) 
     + {\bf p}\cdot({\bf p} + {\bf q}_2) 
     + ({\bf p} + {\bf q}_1)\cdot({\bf p} +  {\bf q}_2)
  \right] 
}{
   (\tilde{\epsilon}^2 - {\bf p}^2) 
   (\tilde{\epsilon}^2 - ({\bf p} + {\bf q}_1)^2)
   (\tilde{\epsilon}^2 - ({\bf p} + {\bf q}_2)^2)
}
\ .
\end{equation}
Here we dropped the term proportional to mixed product
${\bf p} \cdot [{\bf q}_1 \times {\bf q}_2]$. This term 
is antisymmetric under permutation of the two wave vectors
${\bf q}_1$ and~${\bf q}_2$ and therefore does not contribute
to the total density response. 

The integral over momenta can be transformed by employing 
Feynman parameter trick. We employ the identity
\begin{equation}
\frac{1}{ABC} = 2\int\limits_{0}^{1} \int\limits_{0}^{1-x}
\frac{dx dy}{\left[Ax + By + C(1 - x - y)\right]^3}
\ , 
\end{equation}
which can be checked by explicit integration. 
This brings the expression for the quadratic response to the form
\begin{equation}
\Sigma({\bf q}_1, {\bf q}_2)
= -4 \mathop{\rm Im}\nolimits \int\limits_{-\infty}^{\infty}
\int\limits_{0}^{1} \int\limits_{0}^{1-x}
dx dy \frac{d\epsilon}{2\pi}
\int\frac{d^2{\bf p}}{(2\pi)^2}
\frac{
  \tilde{\epsilon}
  \left[\tilde{\epsilon}^2 
      + {\bf p}\cdot({\bf p} + {\bf q}_1)
      + {\bf p}\cdot({\bf p} + {\bf q}_2)
      + ({\bf p} + {\bf q}_1)\cdot({\bf p} +  {\bf q}_2)
   \right]
}{
    \left[\tilde{\epsilon}^2 - {\bf p}^2)
			+ 2 x {\bf p} \cdot{\bf q}_1
			+ 2 y {\bf p} \cdot{\bf q}_2
			+ x {\bf q}_1^2  + y {\bf q}_2^2 \right]^3
}
\ .			      
\end{equation}
The denominator can be rewritten as 
\begin{equation}
\tilde{\epsilon}^2 - ({\bf p} + x {\bf q}_1 + y{\bf q}_2)^2 
+ x(1 - x)  {\bf q}_1^2 + y(1 - y) {\bf q}_2^2
+ 2 xy {\bf q}_1 \cdot{\bf q}_2
\ .
\end{equation}
It is convenient to introduce the new integration variable
\begin{equation}
\tilde{\bf p} = {\bf p} +  x {\bf q}_1 + y{\bf q}_2 
\ .
\end{equation}
This brings the quadratic response to the form
\begin{equation}
\label{U2-before-p}
\Sigma({\bf q}_1, {\bf q}_2)
= -4 \mathop{\rm Im}\nolimits \int\limits_{-\infty}^{\infty}
\int\limits_{0}^{1} \int\limits_{0}^{1-x}
dx dy \frac{d\epsilon}{2\pi}
\int\frac{d^2\tilde{\bf p}}{(2\pi)^2}
\frac{
  \tilde{\epsilon}
  \left[\tilde{\epsilon}^2  + 3\tilde{\bf p}^2 + \tilde{\bf
  p}\cdot{\bf L} + K
  \right]
}{
    \left[\tilde{\epsilon}^2 - \tilde{\bf p}^2 - Q^2 \right]^3
}
\ , 			      
\end{equation}
with~$Q^2 = {\bf q}_1^2 x(1 - x) + {\bf q}_2^2 y(1 - y) + 2{\bf q}_1
\cdot {\bf q}_2 xy$. The quadratic expression in the denominator
represents
\begin{eqnarray}
3{\bf p}^2 + \tilde{\bf p}\cdot{\bf L} + K(\tilde{\bf p}) &=& 
\left(\tilde{\bf p} - {\bf q}_1 x - {\bf q}_2 y\right) 
\cdot\left(\tilde{\bf p} + {\bf q}_1 (1 - x) - {\bf q}_2 y\right)
\nonumber
\\
&+& 
\left(\tilde{\bf p} - {\bf q}_1 x - {\bf q}_2 y\right)
\cdot\left(\tilde{\bf p} - {\bf q}_1 x + {\bf q}_2 (1 - y)\right)
\\
&+&
\left(\tilde{\bf p} + {\bf q}_1 (1 - x) - {\bf q}_2 y\right)
\cdot\left(\tilde{\bf p} - {\bf q}_1 x + {\bf q}_2 (1 - y)\right)
\nonumber
\ .
\end{eqnarray}
We do not give the explicit forms of~$K$ and~${\bf L}$. Later, it
becomes clear that their details are unimportant. We only note
that these quantities are independent of~$\tilde{p}$
and~$\tilde{\epsilon}$.

The integral in Eq.~(\ref{U2-before-p}) can be easily evaulated. 
The term proportional to~$\tilde{\bf p} \cdot {\bf L}$ vanishes 
after integration over angles. Then, 
\begin{equation}
\int\limits_{0}^{\infty} \frac{2\pi \tilde{p} d\tilde{p}}{(2\pi)^2} 
\frac{\tilde{\epsilon}(\tilde{\epsilon}^2 + 3{\tilde p}^2 + K)}{
(\tilde\epsilon^2 - \tilde{p}^2 - Q^2)^3
}
= - \frac{\tilde{\epsilon}}{4\pi}
\int\limits_{Q^2 - \tilde\epsilon^2}^{\infty} 
\frac{dz}{z^3} \left(\tilde\epsilon^2 + K + 3 (z - Q^2 + \tilde\epsilon^2)\right)
\ , 
\end{equation}
where~$z = \tilde{p}^2 + Q^2  - \tilde\epsilon^2$. Calculating the
integral, we find
\begin{equation}
\int\limits_{0}^{\infty} {2\pi \tilde{p} d\tilde{p}}{(2\pi)^2}
\frac{\tilde{\epsilon}(\tilde{\epsilon}^2 + 3{\tilde p}^2 + K)}{
(\tilde\epsilon^2 - \tilde{p}^2 - Q^2)^3
}
= \frac{\tilde\epsilon}{4\pi}
\left[
\frac{4\tilde\epsilon^2 - 3Q^2 + K}{2 (\tilde\epsilon^2 - Q^2)^2}
- \frac{3}{(\epsilon^2 - Q^2)}
\right]
= \frac{\tilde\epsilon}{8\pi}
\frac{-2\tilde\epsilon^2 + 3 Q^2 + K}{(\epsilon^2 - Q^2)^2}
\ .
\end{equation}
Thus, we find for the quadratic response
\begin{equation}
\Sigma({\bf q}_1, {\bf q}_2) 
= - \frac{1}{4\pi^2} \mathop{\rm Im}\nolimits
\int\limits_{0}^{1}dx
\int\limits_{0}^{1-x}dy
\int\limits_{-\infty}^{\infty} 
\frac{\tilde\epsilon d\epsilon (-2\tilde\epsilon^2 + 3 Q^2 + K)}{
   (\epsilon^2 - Q^2)^2}
\ .
\end{equation}
To perform the integration over the electron energy~$\epsilon$, 
we note the following. The integrand can be represented via partial
fractions as
\begin{equation}
\label{U2-partial-fractions}
\frac{\tilde\epsilon (-2\tilde\epsilon^2 + 3 Q^2 + K)}{
    (\epsilon^2 - Q^2)^2}
    = \frac{-2 \tilde\epsilon}{\epsilon^2 - Q^2} 
    + \frac{B\tilde\epsilon}{(\epsilon^2 - Q^2)^2}
\ , 
\end{equation}
where~$B$ is some $\epsilon$-independent quantity. 
The integrals can be calculated by introducing the new variable~$t = 
\tilde\epsilon^2$.
Due to the definition of~$\tilde\epsilon \equiv \epsilon + \mu + 
i 0 \mathop{\rm sgn}\epsilon$, the varialbe~$\tilde\epsilon$
runs over a contour parallel to real axis, slightly above it
for~$\mathop{\rm Re}\nolimits \tilde \epsilon > \mu$, 
and slightly below for~$\mathop{\rm Re}\nolimits \epsilon < \mu$. 
It is then clear that the second term in the
expansion~(\ref{U2-partial-fractions}) becomes zero when integrated 
over such a contour, so that the integral is in fact determined by the
first term only: 
\begin{equation}
\Sigma({\bf q}_1, {\bf q}_2) 
= \frac{1}{2\pi^2} \mathop{\rm Im}\nolimits
\int\limits_{0}^{1}dx
\int\limits_{0}^{1-x}dy
\int\limits_{-\infty}^{\infty} 
\frac{\tilde\epsilon d\epsilon}{
   \epsilon^2 - Q^2}
\ .
%= {1}{4\pi^2} \mathop{\rm Im}\nolimits
%\int\limits_{0}^{1}dx
%\int\limits_{0}^{1}dy
%\int\limits_{-\infty}^{\infty}
%\frac{dt}{t  - Q^2}
%\ .
\end{equation}
This expression formally diverges at large~$\epsilon$. This means that
Eq.~(\ref{U2-initial}) must be regularised to ensure the proper
T-ordering of the operators. This is usually achieved by introducing
convergence factor~$e^{i \epsilon\tau}$, with~$\tau \to +0$. (See 
Abrikosov, Gorkov and Dzyaloshinskii for a detailed discussion.)
This prescription implies that we always close the 
integration contour in the upper half-plane. 

One can now calculate
the integral using the method of residues. The integrand has
two poles at~$\epsilon^2 = \pm Q^2$, each contributing the residue of~$1/2$.
Besides that, closing the integration contour in the upper half-plane
introduces a contribution of a semicircle, $-\pi i$. The contour 
catches only  one pole if~$Q^2 < 0$, or if~$\mu > 0$ and $0 < Q^2 < \mu^2$.
It catches two poles if~$\mu < 0$ and $Q^2 < \mu^2$. Finally, it does
not catch the poles if~$\mu < 0$ and~$Q^2 < \mu^2$. 
Adding the contributions, we find that if~$\mu > 0$ the integral is
nonzero only if $0 < Q^2 < \mu^2$. Otherwise, the contribution of the 
upper semicircle cancels the contribution of one pole, so that the 
integral is equal to~$\pi i$. If, on the other hand, $\mu < 0$, 
the integral is non-zero only if~$0 < Q^2 < \mu^2$, when no poles
are catched by the contour. This gives
\begin{equation}
\Sigma({\bf q}_1, {\bf q}_2) 
= \frac{\mathop{\rm sgn}\mu}{2\pi} 
\int\limits_{0}^{1}dx
\int\limits_{0}^{1 - x}dy 
\theta\left[\mu^2 - Q^2(x, y)\right]
\theta\left[Q^2(x, y)\right]
\ , 
%= {1}{4\pi^2} \mathop{\rm Im}\nolimits
%\int\limits_{0}^{1}dx
%\int\limits_{0}^{1}dy
%\int\limits_{-\infty}^{\infty}
%\frac{dt}{t  - Q^2}
%\ .
\end{equation}
with~$Q^2 = q_1^2 x(1 - x) + q_2^2 y (1 - y)  - 2 {\bf q}_1 \cdot{\bf
q}_2 xy$, as discussed before. 
The step function~$\theta(x)$ is defined as usual: $\theta(x > 0)
= 1$, and~$\theta(x < 0) = 0$. Assuming~$q=0$, one arrives at
the quadratic contribution which can be obtained by Thomas-Fermi
approximation: $\rho = (\mu + U)^2  / 4\pi$.

In fact, one can show that the quantity~$Q^2(x, y)$ is always positive
in the integration domain~$0 \leq y \leq 1 - x$, $0 \leq x \leq 1$. 
Indeed, the quadratic part of this expression is nothing but
$-({\bf q}_1 x + {\bf q}_2 y)^2$, and is therefore negatively 
defined. Thus, the function could have a maximum inside the domain, 
but its minimum is achieved at the boundary. 
At the~$x = 0$ boundary, $Q^2(x, y) = q_1^2 y (1 - y)0$, 
which is positive for~$0 < y < 1$. 
At the~$y = 0$ boundary, $Q^2 = q_2^2 x (1 - x) \geq 0$. 
At the~$y = 1 - x$ boundary, we find~$Q^2 = ({\bf q}_1 + {\bf q}_2)^2
x(1 - x) \geq 0 $. Therefore, one can write
\begin{equation}
\Sigma({\bf q}_1, {\bf q}_2) 
= \frac{\mathop{\rm sgn}\mu}{2\pi} 
\int\limits_{0}^{1}dx
\int\limits_{0}^{1 - x}dy 
\theta\left[\mu^2 - Q^2(x, y)\right]
\ . 
%= {1}{4\pi^2} \mathop{\rm Im}\nolimits
%\int\limits_{0}^{1}dx
%\int\limits_{0}^{1}dy
%\int\limits_{-\infty}^{\infty}
%\frac{dt}{t  - Q^2}
%\ .
\end{equation}
The maximal value of~$Q^2(x, y)$ inside the domain cannot 
exceed~$(q_1 + q_2)^2/4$. Therefore, if~$q_1 + q_2 < 2|\mu|$, 
$\Sigma({\bf q}_1, {\bf q}_2) = 1/4\pi \mathop{\rm sgn}\nolimits$.

At larger~$q$, it quickly falls off. In particular, 
for large values of~$q_{1, 2} \gg \mu$, the quantity~$Q^2 \gg \mu$ 
nearly everywhere except for the corners~$x = y = 0$, $x = 1$ and~$y =
1$. Near these corners,~$Q^2 \sim q^2 (x + y)$. This gives the 
size of the actual integration domain to be~$x_0 \sim y_0 \sim \mu^2 /
q^2$. The area is then~$x_0 y_0 \sim \mu^4/q^4$. Thus, we obtain the
following estimate~$\Sigma (q\gg \mu) \sim \mu^4 / q^4$. 

\end{document}
